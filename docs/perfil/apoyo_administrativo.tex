\documentclass[letterpaper,11pt]{article}
\usepackage[spanish]{babel}
\usepackage[utf8]{inputenc}

\usepackage{rotating}
\usepackage{multirow}

\usepackage{lmodern}
\usepackage[T1]{fontenc}
\usepackage{textcomp}

\usepackage[
pdfauthor={Carlos Eduardo Caballero Burgoa},%
pdftitle={Migración del sistema de apoyo administrativo},%
colorlinks,%
citecolor=black,%
filecolor=black,%
linkcolor=black,%
urlcolor=black
pdftex]{hyperref}

\title{Migración de las funcionalidades del sistema de apoyo administrativo}
\author{Carlos Eduardo Caballero Burgoa}

\begin{document}
\maketitle
\section{Introducción}
Una de las tareas mas pesadas en el laboratorio de desarrollo es el 
mantenimiento de aplicaciones construidas anteriormente, este tipo de tareas en
realidad ocupa mucho del tiempo que se tiene en este laboratorio.

De entre estas, una de las aplicaciones que más importancia posee, es el
sistema de apoyo administrativo, el cual es actualmente utilizado por muchos
departamentos en la facultad de ciencias y tecnología.

Este documento hace mención de los aspectos que se han visto necesarios para
que las funciones actuales puedan mejorarse, e inclusive crearse nuevas
funcionalidades sobre el sistema de apoyo administrativo.

\section{Antecedentes}
En el anterior semestre (2013/I) se ha invertido en el sistema de apoyo
administrativo, aproximadamente 270 horas, alrededor del 70\% del tiempo
total de trabajo. El conjunto de tareas desempeñadas consiste en la corrección
de errores, mantenimientos rutinarios, y apoyo en capacitación de los usuarios.

Toda esta inversión de tiempo, aun irá en crecimiento debido a la migración
constante de los ordenadores de los usuarios, de un sistema operativo Windows
XP de 32 bits, a un sistema operativo Windows 7 de 64 bits. Si bien se ha
intentado en el pasado portar el sistema sin la necesidad de recompilarlo. Este
presenta varios errores en muchas librerías de terceros propias del IDE de
desarrollo, haciendo que la adaptación necesaria requiera mayores gastos de
esfuerzo y tiempo\cite{Delphi1}.

Ademas de las consideraciones referentes a la naturaleza del programa
(aplicación de escritorio con conexión a una base de datos en PostgreSQL), se 
han encontrado varios problemas referentes a la redundancia de datos, es común
para el encargado del mantenimiento del sistema, conectarse por medio de la
aplicación \emph{PgAdmin} para hacer correcciones manuales sobre la
información. Esto causado en parte por la falta de algunas interfaces para la
administración de la información presente en el diseño de base de datos.

Se debe considerar también que no se administra una instancia del sistema, sino
mas bien nueve instancias diferentes de la misma aplicación, una por cada
departamento en el que se encuentra una copia del mismo.

Estos departamentos son:
\begin{itemize}
\item Departamento de Biología.
\item Departamento de Ingeniería Civil.
\item Departamento de Ingeniería Eléctrica.
\item Departamento de Física.
\item Departamento de Ingeniería Industrial.
\item Departamento de Informática.
\item Departamento de Matemáticas.
\item Departamento de Ingeniería Mecánica.
\item Departamento de Química.
\end{itemize}

Otro aspecto a tomar en cuenta, son las perspectivas a futuro que posee el
lenguaje de programación \emph{Delphi}, si bien la empresa original creadora
del IDE para programación de aplicaciones \emph{Delphi} (Borland) a
terciarizado sus cuotas de mercado a la empresa \emph{Embarcadero}, esta ha
mantenido el soporte, ademas de mantener mucha de la compatibilidad hacia atrás
que se requeriría para la actualización del sistema\cite{Delphi2}.

\section{Definición del problema}
A la fecha el aprendizaje del lenguaje de programación Delphi, es mas
dificultoso, de un tiempo a esta parte el numero de personas que conocen este
lenguaje se ha reducido. Como consecuencia de esto la curva de aprendizaje y
el grado de especialización que un programador puede adquirir se ha visto
dramáticamente reducido. El efecto inmediato de este antecedente se refleja en
los tiempos de respuesta a problemas en la aplicación, que tiende a ser mayor
que lo normal, ademas de la alta dificultad para el planteamiento de nuevas
funcionalidades, o la corrección de errores emergentes.

Otro punto que llama la atención es la perspectiva con la que los usuarios
manejan el sistema en los diferentes departamentos, la mayor parte de los
usuarios, solo utilizan un conjunto muy limitados de funciones, particularmente
me refiero a las funciones para la impresión de seguimientos y nombramientos.

Algo que resulta un inconveniente constante, es la preocupante desorganización de la documentación disponible del sistema de apoyo administrativo, haciendo que el tiempo invertido en el mantenimiento sea mayor del necesario.

Por lo mencionado se define el problema como:

\emph{“La inversión de tiempo y esfuerzo que el sistema de apoyo administrativo
requiere, no justifica su manutención en el lenguaje de programación y
arquitectura que poseen actualmente.”}

\section{Objetivos}

\subsection{Objetivo General}
Reescribir el conjunto de funciones encargadas de la realización de
seguimientos y nombramientos en el sistema de apoyo administrativo, en un
lenguaje de programación que aplique al contexto actual.

\subsection{Objetivos Específicos}
\begin{itemize}
\item Buscar una alternativa mejor al lenguaje de programación Delphi, basado
      en el contexto que sería deseable para las condiciones actuales de
      trabajo, de modo que agilice la curva de aprendizaje, y reduzca los
      conocimientos necesarios para un desarrollador en el laboratorio de
      desarrollo.
\item Recopilar todos los volúmenes documentales que posea el sistema de apoyo
      administrativo, para poder reestructurar las intenciones iniciales que se
      hayan tenido al momento de escribir el sistema original.
\item Centralizar toda la información que se administra de la facultad de
      ciencias y tecnología en un solo sistema que pueda proveer una gran
      variedad de servicios adicionales.
\end{itemize}

\section{Ingeniería de proyecto}
Véase el cuadro \ref{ingenieriadeproyecto} en la página
\pageref{ingenieriadeproyecto}.

\begin{sidewaystable}
\centering
\small
\begin{tabular}{|l|l|l|p{6.5cm}|l|}
\hline
Objetivo General & Problema & Objetivos Específicos & Actividades &
Resultados \\
\hline
\multirow{9}{3.0cm}{Reescribir el conjunto de funciones encargadas de la
realización de seguimientos y nombramientos en el sistema de apoyo
administrativo, en un lenguaje de programación que aplique al contexto
actual.} &
\multirow{3}{4cm}{El aprendizaje del lenguaje de programación Delphi, es mas
dificultoso, de un tiempo a esta parte el numero de personas que conocen este
lenguaje se ha reducido.} &
\multirow{3}{3.5cm}{Buscar una alternativa mejor al lenguaje de programación
Delphi, de modo que agilice la curva de aprendizaje, y reduzca los
conocimientos minimos necesarios.} &
Recopilar la información necesaria sobre el lenguaje de programación Delphi,
ademas de las ventajas y desventajas que posee este lenguaje. &
\multirow{3}{2.5cm}{Documento de especificación de requerimientos.} \\
\cline{4-4}
& & & Recopilar los nuevos requerimientos no funcionales deseables para el
sistema, y contrastarlos con los originales. & \\
\cline{4-4}
& & & Analizar las posibles alternativas al desarrollo con el lenguaje de
programación Delphi, basadas en el contexto necesario. & \\
\cline{2-5}
& \multirow{3}{3cm}{Notoria desorganización de la documentación disponible del
sistema de apoyo administrativo, haciendo que el tiempo invertido en el
mantenimiento sea mayor del necesario. } &
\multirow{3}{3.5cm}{Recopilar todos los volúmenes documentales que posea el
sistema de apoyo administrativo, para poder reestructurar las intenciones
iniciales que se hayan tenido al momento de escribir el sistema original.} &
Agrupar toda la información que existe sobre el sistema de apoyo 
administrativo, revisar a fondo la importancia y calidad de la documentación. &
\multirow{3}{2.5cm}{Información documental correctamente organizada.} \\
\cline{4-4}
& & & Analizar la información que se tiene disponible sobre el sistema de apoyo
administrativo, de modo que toda redundancia pueda ser descubierta. & \\
\cline{4-4}
& & & Creación de un documento base para referencia de toda documentación
existente, de modo que minimice los problemas de ambigüedad de la organización
de la documentación. & \\
\cline{2-5}
& \multirow{3}{3cm}{Los gastos de tiempo y esfuerzo invertidos en el
mantenimiento en el sistema de apoyo administrativo, tienden a incrementarse dramáticamente. } &
\multirow{3}{3.5cm}{Centralizar toda la información que se administra de la
facultad de ciencias y tecnología en un solo sistema que pueda proveer una gran
variedad de servicios adicionales.} &
Analizar la base de datos, todas las ventajas y desventajas que posea el modelo
actual. &
\multirow{3}{2.5cm}{Paquete de software con las funcionalidades de seguimiento
y nombramiento de docentes y auxiliares.} \\
\cline{4-4}
& & & Hacer todas las correcciones que se vean convenientes en la
estructuración de la información. & \\
\cline{4-4}
& & & Portar todo el sistema de apoyo administrativo en el lenguaje de
programación que se ha decidido. & \\
\hline
\end{tabular}
\caption{Ingeniería de proyecto}
\label{ingenieriadeproyecto}
\end{sidewaystable}

\section{Justificación}
Ademas del ahorro de tiempo para el laboratorio de desarrollo que implicaría
refactorizar el sistema de apoyo administrativo. Existen algunos problemas
inherentes con la programación en \emph{Delphi} que podrían corregirse, estos
son:

\begin{itemize}
\item Trabajo sobre la codificación de caracteres UTF-8, en lugar de la
      codificación ASCII, que es utilizada en \emph{Delphi}.
\item Creación de datos abiertos. Para la construcción de funcionalidades
      adicionales desacopladas.
\item Incremento de las funcionalidades, a partir de una arquitectura adaptada
      a las nuevas condiciones del contexto actual.
\end{itemize}

\section{Alcance}

En esta primera etapa únicamente se considera cubrir las funcionalidades de
nombramiento y seguimiento de docentes y auxiliares. Se dejan las otras
funcionalidades para una posterior evaluación.

\begin{thebibliography}{99}

\bibitem{Delphi1} Migrating Delphi Code: Technical Challenges.\\
Extraído el 23 de Septiembre del 2013, de\\
http://blog.marcocantu.com/blog/migrating\_delphi\_code\_tech\_challenges.html.

\bibitem{Delphi2} What's the future of Delphi?\\
Extraído el 24 de Septiembre del 2013, de\\
http://chriskulbacki.com/whats-future-delphi.

\end{thebibliography}

\end{document}
