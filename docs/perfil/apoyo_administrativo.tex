\documentclass[letterpaper,11pt]{article}
\usepackage[spanish]{babel}
\usepackage[utf8]{inputenc}

\usepackage{rotating}
\usepackage{multirow}

\usepackage{lmodern}
\usepackage[T1]{fontenc}
\usepackage{textcomp}

\usepackage[
pdfauthor={Carlos Eduardo Caballero Burgoa},%
pdftitle={Migración del sistema de apoyo administrativo},%
colorlinks,%
citecolor=black,%
filecolor=black,%
linkcolor=black,%
urlcolor=black
pdftex]{hyperref}

\title{Migración de las funcionalidades del sistema de apoyo administrativo}
\author{Carlos Eduardo Caballero Burgoa}

\begin{document}
\maketitle
\section{Introducción}
Una de las tareas mas pesadas en el laboratorio de desarrollo es el 
mantenimiento de aplicaciones construidas anteriormente, este tipo de tareas en
realidad ocupa mucho del tiempo que se tiene en este laboratorio.

De esas, una de las aplicaciones que más importancia posee, es el sistema de
apoyo administrativo, 

\section{Antecedentes}
En el anterior semestre (2013/I) se ha invertido en el sistema de apoyo
administrativo, aproximadamente 270 horas, alrededor del 70\% del tiempo
total de trabajo. El conjunto de tareas desempeñadas consiste en la corrección
de errores, mantenimientos rutinarios, y apoyo en capacitación de los usuarios.

Toda esta inversión de tiempo, aun irá en crecimiento debido a la migración
constante de los ordenadores de los usuarios, de un sistema operativo Windows
XP de 32 bits, a un sistema operativo Windows 7 de 64 bits. Si bien se ha
intentado en el pasado portar el sistema sin la necesidad de recompilarlo. Este
presenta errores en las librerías propias del sistema operativo, haciendo que
la adaptación necesaria requiera mayores gastos de esfuerzo y tiempo
\cite{Delphi1}.

Ademas de las consideraciones referentes a la naturaleza del programa
(aplicación de escritorio con conexión a una base de datos en PostgreSQL), se 
han encontrado varios problemas referentes a la redundancia de datos, es común
para el encargado del mantenimiento del sistema, conectarse por medio de la
aplicación PgAdmin III para hacer correcciones manuales sobre la información.
Esto causado en parte por la falta de algunas interfaces para la administración
de la información presente en el diseño de base de datos.

Se debe considerar tambien que no se administra una instancia del sistema, sino
mas bien nueve (9) instancias diferentes de la misma aplicación, uno por cada
departamento en el que se encuentra una copia del mismo.

Se ha observado ademas que la conexión entre la aplicación de escritorio y el
manejador de base de datos se lleva a cabo mediante un protocolo de
comunicaciones sin encriptación.

Otro aspecto a tomar en cuenta, son las perspectivas a futuro que posee el
lenguaje de programación Delphi, si bien la empresa original creadora del IDE
para programación de aplicaciones Delphi (Borland) a terciarizado sus cuotas de
mercado a la empresa \emph{Embarcadero}.

\section{Definición del problema}



Por lo mencionado se define el problema como:

\emph{“La inversión de tiempo y esfuerzo que el sistema de apoyo administrativo
requiere, no justifica su manutención en el lenguaje y arquitectura que poseen
actualmente.”}

\section{Objetivos}
Reescribir el conjunto de funciones encargadas de la realización de
seguimientos y nombramientos en el sistema de apoyo administrativo, en un
lenguaje de programación orientado al desarrollo web, mas que al desarrollo de
aplicaciones de escritorio.

\subsection{Objetivo General}

\subsection{Objetivos Específicos}
\begin{itemize}
\item .
\end{itemize}

\section{Ingeniería de proyecto}
Véase el cuadro \ref{ingenieriadeproyecto} en la página \pageref{ingenieriadeproyecto}.

\begin{sidewaystable}
\centering
\small
\begin{tabular}{|l|l|l|p{6.5cm}|l|}
\hline
Objetivo General & Causa & Objetivos Específicos & Actividades & Resultados \\
\hline
\multirow{15}{2.6cm}{Promover la interacción académica entre los estudiantes mediante el uso de
una red social para mejorar los métodos de adquisición del conocimiento.} &
\multirow{3}{3cm}{Pérdida de tiempo al crear espacios de interacción por parte de los docentes 
y estudiantes.} &
\multirow{3}{3.5cm}{Agilizar la creación de espacios virtuales para incrementar la cantidad y
variabilidad de estos.} &
Analizar las herramientas actuales para creación de espacios virtuales. &
\multirow{3}{2.5cm}{Modulo para la creación de espacios virtuales.} \\
\cline{4-4}
& & & Diseñar e implementar un método eficaz y cómodo para la manipulación de los espacios virtuales. & \\
\cline{4-4}
& & & Evaluar la usabilidad y amigabilidad de la herramienta desarrollada. & \\
\cline{2-5}
& \multirow{3}{3cm}{De\-sa\-pro\-ve\-cha\-mi\-en\-to del conocimiento y experiencia de los estudiantes.} &
\multirow{3}{3.5cm}{Facilitar el intercambio de recursos entre los estudiantes para acelerar la
adquisición de experiencia.} &
Diseñar e implementar de una arquitectura orientada a la provisión de recursos web. &
\multirow{3}{2.5cm}{Modulo para la publicación de recursos web multimedia.} \\
\cline{4-4}
& & & Diseñar e implementar de una arquitectura orientada al consumo de recursos web. & \\
\cline{4-4}
& & & Crear un API funcional para el acceso a la información. & \\
\cline{2-5}
& \multirow{3}{3cm}{Inadecuada conectividad entre las distintas plataformas web 2.0.} &
\multirow{3}{3.5cm}{Facilitar el intercambio de recursos entre distintas instancias de la red
para mejorar la disponibilidad de recursos.} &
Modelar un proceso de intercambio de información basado en servicios web en el sistema. &
\multirow{3}{2.5cm}{Modulo de conectividad entre instancias de la red social.} \\
\cline{4-4}
& & & Implementar estándares de servicios web en el sistema. & \\
\cline{4-4}
& & & Crear un API funcional para el intercambio de información. & \\
\cline{2-5}
& \multirow{3}{3cm}{Innecesaria sobrecarga de trabajo de los docentes en la atención a los estudiantes.} &
\multirow{3}{3.5cm}{Mejorar los canales de comunicación entre estudiantes y docentes para facilitar
la retroalimentación.} &
Crear espacios virtuales para las respectivas materias, grupos y dinámicas de grupo. &
\multirow{3}{2.5cm}{Módulos auxiliares para el apoyo de la enseñanza en aula.} \\
\cline{4-4}
& & & Diseñar e implementar métodos que mejoren los canales de comunicación. & \\
\cline{4-4}
& & & Crear indicadores para el seguimiento de la fluidez de comunicación en el sistema. & \\
\cline{2-5}
& \multirow{3}{3cm}{Progresiva pérdida del interés de parte de los estudiantes.} &
\multirow{3}{3.5cm}{Planear estrategias que fomenten la participación para mantener activo el sistema.} &
Revisar los métodos actuales de atracción de los usuarios en los sistemas multiusuario. &
\multirow{3}{2.5cm}{Módulos auxiliares de fomento a la participación de los usuarios.} \\
\cline{4-4}
& & & Diseñar e implementar los módulos definidos. & \\
\cline{4-4}
& & & Crear indicadores para el seguimiento del grado de interés en el sistema. & \\
\hline
\end{tabular}
\caption{Ingeniería de proyecto}
\label{ingenieriadeproyecto}
\end{sidewaystable}

\section{Justificación}

Ademas del ahorro de tiempo para el laboratorio de desarrollo que implicaría
refactorizar el sistema de apoyo administrativo. Existen algunos problemas
inherentes con la programación en Delphi que podrían corregirse, estos son:

\begin{itemize}
\item Trabajo sobre la codificación de caracteres UTF-8, en lugar de la
	  codificación ASCII, que es utilizada en Delphi.
\item Creación de datos abiertos. Para la construcción de funcionalidades
	  adicionales desacopladas.
\end{itemize}

\section{Innovación tecnológica}

\section{Alcance}

\begin{thebibliography}{99}

\bibitem{Delphi1} Migrating Delphi Code: Technical Challenges.\\
Extraído el 23 de Septiembre del 2013, de\\
http://blog.marcocantu.com/blog/migrating\_delphi\_code\_tech\_challenges.html.

\end{thebibliography}

\end{document}
